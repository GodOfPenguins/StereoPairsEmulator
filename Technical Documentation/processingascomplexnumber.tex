\subsection{Processing as a Complex Number}

The transfer function that a virtual microphone applies to the signal is applied in the time and amplitude domains. This can be represented as a complex number in the form:

\begin{equation}\label{vMicFunction}
m(S) = \Delta{}v + \Delta{}t
\end{equation}

Where $\Delta v$ is the change in amplitude, and $\Delta t$ is the change in time

\subsection{$\Delta{}v$ Calculation}

The $\Delta{}V$ component of the function $m(S)$ if based off of the positional relationships of the virtual microphone and the virtual sound source, and the polar attenuation pattern setting of the microphone. The transfer function defined by this relationship can be found as a function of $\theta_{mS}$, $v(\theta_{mS})$. Recalling the polar pattern equation from (\ref{polarPatternPlot}), a virtual microphone can be seen as consisting of two components: a nondirectional component, and a bidirectional component\footnote{This mirrors the construction of some types of real directional microphones.}. Following this, by replacing $\phi$ in (\ref{polarPatternPlot}) with $\theta_{mS}$, the resulting function, $m(\theta_{mS}, p_m)$, can be used to find the amplitude scalar that represents the magnitude of the effect of the virtual microphone's polar attenuation pattern on the source signal, $S$, when $S$ is at the relative angle represented by $\theta_{mS}$.

Therefore, $\Delta{}v$ can be found as the product of the source signal and the angle-dependent scalar coefficient:

\begin{equation}
\Delta{}v_m = S \cdot m(\theta_{mS}, p_m)
\end{equation}

Which can rewritten more explicity as:

\begin{equation}
\Delta{}v_m = S\left[ \left(1 - p_m\right) + p_m\cos(\theta_{mS}) \right]
\end{equation}


\subsection{$\Delta{}t$ Calculation}

The $\Delta{}t$ component represents the time that it takes for the sound emitted by source $S$ to reach the position of virtual microphone $m$. Following from (\ref{distance}), the value of $d_{\vec{mS}}$ when combined with the speed of sound constant $c$ can be used to find the time-domain component.

\subsubsection{Calculating the speed of sound}

The speed of sound in real spaces is generally considered to be ~\SI[per-mode=fraction]{343}{\m\per\s}. 

\subsection{$M(S)$ Array processing}

It then follows that the complex-number formulation of the function $m(S)$ can be written out:

\begin{equation}
	m(S) = S\left[\left(1 - p_m\right) + p\cos(\theta_{mS})\right] + \hat{t}\frac{d_{\vec{v}}}{343 + \Delta{}c}
\end{equation}

Which is comprised of two components: the adjusted amplitude of $S$, and the time-domain shift of $S$.

The virtual microphone function (\ref{vMicFunction}) can be applied for each element in $M$ to find the effect of the total microphone array:

\begin{equation}
M(S) = \{m_1(S) ... m_n(S)\}
\end{equation}

The elements of $M(S)$ can be apportioned to the mixdown channels following:

\begin{equation}\label{output}
\begin{bmatrix} L(M(S)_n) \\ R(M(S)_n) \end{bmatrix} = gM(S)_n \begin{bmatrix} k_L \\ k_R \end{bmatrix}
\end{equation}

Where $g$ is a scalar constant, and $k_L$ and $k_R$ are related proportionality constants to determine the level of the element $M(S)_n$ within the encoded left and right channels. 

\subsubsection{Output implementation}

For purposes of this implementation, $g$ is a scalar value applied per pair (or to the center microphone). The proportionality constants $k$ for the flanks follow a standard sine-cosine panning function. The center microphone uses $k = 1$.

The $g$ scalar is exposed to the user as a value $\alpha$, where $\alpha$ is a number in decibels and $\alpha = [-20,0]$.

Converting $\alpha_m$ to a scalar $g_m$ can be done:

\begin{equation}\label{dbscalarconvert}
g_m = 10^{\frac{\alpha}{20}}
\end{equation}

For the apportionment of the signal to the mixdown channels, the user inputs a value, $\beta$, where $\beta = [0, 1]$. The value for $\beta$ is used to indicate the separation of the virtual microphone pair into their corresponding mixdown channels. Lower values of $\beta$ indicate that both microphones in the pair should increasingly come out of both mixdown channels; whereas higher values indicate more separation into their corresponding channel. The relationship with $k$ can be shown as:

\begin{equation}\label{pairsPanning}
\begin{bmatrix} k_1 \\ k_2 \end{bmatrix} = \begin{bmatrix} \sin{k(\beta)} \\ \cos{k(\beta)} \end{bmatrix} \; | \; k(\beta) = \beta\frac{pi}{4} + \frac{pi}{4}
\end{equation}

Following this, for any virtual microphone pair in $M$, the placement of the encoded sounds within the output mixdown follows:

\begin{equation}\label{centerPanning}
\begin{bmatrix} y(L) \\ y(R) \end{bmatrix} = \begin{bmatrix} \sin{k(\beta)} & \cos{k(\beta)} \\ \cos{k(\beta)} & \sin{k(\beta)} \end{bmatrix} \cdot g\begin{bmatrix} m(S)_L \\ m(S)_R \end{bmatrix}
\end{equation}

While $m(S)_{center}$ is simply:

\begin{equation}
\begin{bmatrix} y(L) \\ y(R) \end{bmatrix} = gm(S)_{c}J_{2,1}
\end{equation}

Thus, the final, encoded output of the processor can be represented as:

\begin{equation}\label{output}
y(M, S) = \sum\limits_{i=1}^{|M|} y(M(S)_i)
\end{equation}

Which, when taken with the relationship to the UI in (\ref{pairsPanning}) and (\ref{centerPanning}), (\ref{output}) can be expanded to:

\begin{equation}
y(M, S) = \begin{bmatrix} y(L)_\text{mains} \\ y(R)_\text{mains} \end{bmatrix} + \begin{bmatrix} y(L)_\text{flanks} \\ y(R)_\text{flanks} \end{bmatrix} + \begin{bmatrix} y(L)_\text{center} \\ y(R)_\text{center} \end{bmatrix}
\end{equation}
