Audio localization in stereophonic recording focuses on balancing the pressure level and time difference cues that inform audio localisation through interaural time differences (ITD) and interaural level differences (ILD). Within the stereophonic recording praxis, it is understood that the weighting of ITD and ILD cues contributes to varying desireable qualities within the resultant soundfield. Stereophonic microphone arrays are also chosen to account for the desired balance of direct sound and reflected sound and to accomodate the recording angle of the perceptual "soundstage" and translate it to audio reproduction systems with minimal angular distortion.

Within the context of localisation, it is understood that a relative dominance of time-domain cues creates a greater percieved sense of stereophonic width/envelopment, whereas a greater relative preponderance of level-based cues increases the sense of locative precision. Microphones, as pressure transducers, encode local changes in atmospheric pressure into changes in electrical pressure, and can be practically thought of as sampled points in space. Additionally, microphones have varying types of directivity -- represented by a polar plot -- that represent how efficiently they transduce sound based on the sound-wave angle of arrival relative to the microphone's orientation. For directional microphones with cardioid-style polar patterns, the amount of attenuation tends to increase as the sound-wave's direction of arrival moves further away from the oriented "front" of the microphone\footnote{There is some complexity with this as the directivity pattern value increases past a pure-cardioid, with an inverse-phase area beginning to present at the rear of the microphone and increasing until the directivity approches a bidirectional pattern}.

By positioning two directional microphones in the same location and facing in two different directions, any sound-wave captured by them will be increasingly less attenuated as it approaches the front of one microphone and more attenuated as it approches the other. If the sound wave approaches from an angle that equiangular to both microphones, then it will be equally attenuated. This equal attenuation, during reproduction, will have the perceptual effect of placing the sound at the center of the virtual sound-stage. 

Similarly, by positioning two nondirectional microphones in to locations within the space, the distance between them will create differences in the time of arrivale based on the speed of propagation of the  sound-wave (the speed of sound). As the direction of arrival shifts away from being perpendicular to an imaginary line drawn from one microphone to the other, there will be an increasing time delay between when the sound is captured in the closer microphone and when it is captured in the further microphone. As with the level-based cues, when the time of arrival is equal (i.e. there is no delay in one microphone) then the perceptual effect during reproduction is one of the sound being centered in the perceptual soundstage. As the delay increases in one microphone, the sound will appear to come from the direction in which it arrives first\footnote{This phenomenon is known as the "precendence effect"}.